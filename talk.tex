\documentclass[onlymath]{beamer}
\usepackage{fontspec}
\usepackage{xunicode,xltxtra}
\usepackage{polyglossia}
\usepackage{xltxtra}
\usepackage{fancybox}
\usepackage{graphicx}

\usepackage{tikz}
\usetikzlibrary{calc,through,positioning,patterns,decorations.markings,shapes}
\tikzset{dot/.style={circle,fill=black,scale=0.5}}
\usepackage{pgfplots}

\usepackage{listings}
\lstset{language=[LaTeX]TeX, basicstyle=\scriptsize\ttfamily, escapechar=|, frame=single}

\newcommand{\abs}[1]{\left \lvert{#1}\right \rvert}
\newcommand\tw\textwidth
\newcommand\neword\emph
\newcommand\code\texttt
\renewcommand\epsilon\varepsilon
\newcommand{\cenfig}[2]{\begin{figure}\centering\includegraphics[scale=#1]{#2}
  \end{figure}}
  
\usefonttheme{professionalfonts}
\usetheme[secheader]{Boadilla}
\usecolortheme{whale}

\setsansfont[Mapping=tex-text]{Myriad Pro}
\setmonofont[Mapping=tex-text]{DejaVu Sans Mono}
\setmainlanguage{russian}

\title{Технические иллюстрации в LaTeX с PGF/TikZ}
\author{Дмитрий Джус}
\institute{АК-101}
\date{16 февраля 2011}

\begin{document}

\begin{frame}
  \titlepage
\end{frame}

\begin{frame}
  \frametitle{План}
  \tableofcontents
\end{frame}

\section{Система вёрстки LaTeX}

\begin{frame}[fragile]
  \frametitle{За что мы любим \LaTeX{}}
  \begin{columns}
    \column{.4\textwidth}
\begin{lstlisting}
% преамбула
\documentclass{article}
\usepackage[utf8]{inputenc}
\usepackage[russian]{babel}
\begin{document}
% документ
  \begin{equation}
  \frac{\pi}{2} = \sum {a_i}
  \end{equation}
\end{document}
\end{lstlisting}
    \column{.55\textwidth}
    \begin{itemize}
    \item Свободная и популярная система вёрстки разных печатных
      материалов, доступная для разных ОС

    \item Документы хранятся в текстовых файлах, которые содержат
      специальную разметку

    \item Из исходных файлов компилируются PDF-документы, пригодные
      для просмотра или печати
    \end{itemize}
  \end{columns}
\end{frame}

\begin{frame}
  \frametitle{Возможности}
  \begin{itemize}
  \item Возможности для прекрасной типографики
  \item Разметка математических формул
    \begin{equation}
      \frac{\pi^4}{2}\int_{-\infty}^\infty{\abs{f(x^{k+1})-f(x^k)}}\, dx=\sum_{\lambda_i^+>0} \sqrt{\left\{\xi_{i,k}^2
        \lambda_i^+ [1-R^2(\lambda_i^+)]\right\}}
    \end{equation}
  
  \item Инструменты структурирования текста для последующего создания
    оглавлений и перекрёстных ссылок на главы, формулы, рисунки

  \item Поощряется стилевое оформление документа

  \item Язык разметки легко дополняется собственными конструкциями
    различного назначения

  \item Вёрстка материалов различных форматов и назначений (статьи,
    книги, презентации)

  \item Огромное количество пакетов расширения
  \end{itemize}
\end{frame}

\section{Пакет расширения PGF/TikZ}
\begin{frame}
  \frametitle{Проблемы иллюстрирования}
  \begin{itemize}
  \item Проблемы с растровыми картинками при масштабировании
  \item Создание графики в сторонних программах затрудняет внедрение
    иллюстраций
  \item Чужеродные текстовые подписи
  \item Технические иллюстрации со сложной геометрической структурой
    не всегда удобно создавать вручную
  \end{itemize}
\end{frame}

\begin{frame}
  \frametitle{PGF/TikZ}
  \begin{itemize}
  \item Позволяет \emph{описывать} иллюстрации прямо в теле
    LaTeX-документа с помощью мини-языка

  \item Набор графических примитивов (точка, линия, окружность,
    кривая)

  \item Различные координатные системы, в том числе связанные с
    элементами иллюстрации

  \item Выполнение геометрических операций вроде поиска пересечения
    линий или касательной к окружности

  \item Глубокие возможности по настройке внешнего вида элементов
    (форма линий, штриховка областей, стрелки)
  \end{itemize}
\end{frame}

\begin{frame}[fragile]
  \frametitle{Пример 1}
  \begin{figure}[!h]
    \centering
    \begin{tikzpicture}
      \coordinate [] (O) at (0, 0) {};
      \coordinate [] (X) at (1, 0.5) {};
      \draw[thick, dashed] (O) -- (X)
       node[sloped, above, pos=0.5] {$r < \epsilon$};
      \node[draw, thick, circle through=(X)] (circle) at (O) {};
    \end{tikzpicture}
  \end{figure}
\begin{lstlisting}
    \begin{tikzpicture}
      % координаты
      \coordinate (O) at (0, 0) {};
      \coordinate (X) at (1, 1.5) {};

      % отрезок
      \draw[thick, dashed] (O) -- (X)
      % подпись
       node[sloped, above] {$r < \epsilon$};

      % окружность
      \node[draw, thick, dashed, circle through=(X)] (circle) at (O) {};
    \end{tikzpicture}
\end{lstlisting}
\end{frame}

\begin{frame}[fragile]
  \frametitle{Пример 2}
    \begin{figure}[!h]
    \centering
    \begin{tikzpicture}
      % узлы
      \coordinate (O) at (-2, 0) {};
      \node[draw, shape=circle, dashed] (A1) [] at (O) {$A_1$};
      \node[draw, shape=rectangle] (A2) [] at ($ (O) + (1, 1) $) {$A_2$};
      \node[draw, shape=star, thick] (A3) [] at ($ (O) + (170:5) $) {$A_3$};

      \draw[->] (A1.west) -- (A3.east);
      \draw[thick, <->] (A1.north) to[out=90, in=150] (A2.west);
    \end{tikzpicture}
  \end{figure}
\begin{lstlisting}
\begin{tikzpicture}
  % узлы
  \coordinate (O) at (-2, 0) {};
  \node[draw, shape=circle, dashed] (A1) at (O) {$A_1$};
  \node[draw, shape=rectangle] (A2) at ($ (O) + (1, 1) $) {$A_2$};
  \node[draw, shape=star] (A3) at ($ (O) + (170:5) $) {$A_3$};

  \draw[->] (A1.west) -- (A3.east);
  \draw[thick, <->] (A1.north) to[out=90, in=150] (A2.west);
\end{tikzpicture}
\end{lstlisting}
\end{frame}

\begin{frame}[fragile]
  \frametitle{Пример 3: модель Клейна в круге}
  \begin{figure}[thb]
  \centering
  \begin{tikzpicture}[scale=3]
    \coordinate [] (O) at (0, 0) {};

    \coordinate [label=above right:$X$] (X) at (60:1) {};
    \coordinate [label=below left:$Y$] (Y) at (200:1) {};

    \coordinate [label=above left:$A$] (A) at ($ (Y)!.3!(X)$) {};
    \coordinate [label=above left:$B$] (B) at ($ (X)!.2!(Y)$) {};

    \coordinate [label=below:$M$] (M) at (280:1) {};
    \coordinate [label=below right:$N$] (N) at (330:1) {};
    
    \coordinate [label=above left:$P$] (P) at (intersection of X--M and Y--N) {};
    
    % disk model
    \node [draw,thick,circle through=(X)] (circle) at (O) {};

    % chords
    \draw (X) --node[auto]{$l$} (Y);
    \draw (X) -- (M);
    \draw (Y) -- (N);

    % a bundle of parallel lines
    \begin{scope}
      \clip (O) circle(1);
      \foreach \a in {0,...,6} \draw[very thin]
                                ($ (P)+(180+\a*12:2) $) -- 
                                ($ (P)+(\a*12:2) $);
    \end{scope}

    %dot marks
    \foreach \p in {X, Y, A, B, P, M, N} \draw node[dot] at (\p) {};
  \end{tikzpicture}
\end{figure}
\end{frame}


\begin{frame}[fragile]
  \frametitle{Пример 4: связь моделей Клейна и Пуанкаре}
  \begin{figure}[!hbt]
  \centering
  \begin{tikzpicture}[scale=1.6]
    \coordinate (O) at (0, 0) {};
    
    % Klein model points
    \coordinate (P) at (-1.42, .04) {};
    \coordinate (Q) at (-0.933,-0.48) {};

    % Projections onto sphere
    \coordinate (Ps) at (-1.42,-.85) {};
    \coordinate (Qs) at (-0.93,-1.3) {};

    % Sphere poles
    \coordinate [label=above:$N$] (N) at (0, 2) {};
    \coordinate (S) at (0, -2) {};

    % Disk control points
    \coordinate [] (L) at ($ (O)-(2,0) $) {};
    \coordinate [] (R) at ($ (O)+(2,0) $) {};

    % Projection lines
    \draw (P)--(Ps);
    \coordinate (Pp)
                at (intersection of Ps--N and O--P);                
    \draw (Ps)--(Pp);
    \draw (Q)--(Qs);
    \coordinate (Qp)
                at (intersection of Qs--N and O--Q);                
    \draw (Qs)--(Qp);

 
    % Fade lower hemisphere                 
    \path [fill=white, opacity=.5]
                (L) to[out=270,in=180] (S)
                    to[out=0,in=270] (R)
                    (O) ellipse(2 and 1);

     % Klein disk
    \draw [opacity=.3, pattern=north east lines]
               (O) ellipse(2 and 1);
    \draw [thick] (O) ellipse(2 and 1);

    % Chord
    \coordinate [dot] (X) at (-1.8,0.4358) {};
    \coordinate [dot] (Y) at (-0.464,-0.97) {};
    \draw[] (X)--(Y);
    \draw[] (X) to[out=-15,in=135,looseness=.9] (Pp)
                to [out=315,in=120,looseness=.7] (Qp)
                to[out=300,in=65,looseness=.8] (Y);
                    
    % Visible projection lines
    \draw (Pp)--(N);
    \draw (P)--(O);
    \draw (Qp)--(N);
    \draw (Q)--(O);

    % Sphere
    \draw[thick] (O) circle(2);

    % labels placed last
    \begin{scope}[inner sep=2pt]
      \node [label=below:$X$] at (X){};
      \node [label=below:$Y$] at (Y){};
      \node [label=right:$O$] at (O){};
      \node [label=left:$P$] at (P){};
      \node [label=left:$Q$] at (Q){};
      \node [label=above:$P^\prime$] at (Pp){};
      \node [inner sep=4pt,label=right:$Q^\prime$] at (Qp){};
      \node [label=below:$P_s$] at (Ps){};
      \node [label=below:$Q_s$] at (Qs){};
    \end{scope}
    
    % dot marks
    \foreach \p in {O,P,Pp,Q,Qp,N} \draw node[dot] at (\p) {};
    \node[dot,color=gray] at (Ps) {};
    \node[dot,color=gray] at (Qs) {};
  \end{tikzpicture}
\end{figure}
\end{frame}

\section{Графики вместе с pgfplots}
\begin{frame}
  \frametitle{Мощный пакет для графиков}

  \begin{itemize}
  \item Является расширением PGF/TikZ
  \item Позволяет строить графики поточечно, по аналитическому
    выражению или из файла
  \item Двух- и трёхмерные графики, различные стили представления
  \end{itemize}

\end{frame}

\begin{frame}[fragile]
  \frametitle{Пример}
   \begin{figure}[!h]
    \centering
    \begin{tikzpicture}
      \begin{axis}[
        y=.3cm,
        ytick={-3.14, 0, 3.14},
        yticklabels={$-\pi$, $0$, $\pi$}]
        \addplot[dashed, draw=black] coordinates{(-5,-1) (-3.4,2) (2,3)};
        \addplot[thick, draw=black, samples=50] {pi*sin(deg(x))};
      \end{axis}
    \end{tikzpicture}
  \end{figure}
\begin{lstlisting}
\begin{tikzpicture}
  \begin{axis}[
    y=.3cm,
    ytick={-3.14, 0, 3.14},
    yticklabels={$-\pi$, $0$, $\pi$}]
    % точки
    \addplot[dashed, draw=black] coordinates{(-5,-1) (-3.4,2) (2,3)};
    % функция
    \addplot[thick, draw=black, samples=50] {pi*sin(deg(x))};
  \end{axis}
\end{tikzpicture}
\end{lstlisting}
\end{frame}

\begin{frame}[fragile]
  \frametitle{Пример: чтение файла}
   \begin{figure}[!h]
    \centering
    \begin{tikzpicture}
      \begin{axis}[y=2cm]
        \addplot file {pts.txt};
      \end{axis}
    \end{tikzpicture}
  \end{figure}
  \begin{itemize}
  \item 
\begin{lstlisting}[linewidth=.8\textwidth]
        \addplot file {pts.txt};
\end{lstlisting}
    
  \item Файл содержит два ряда точек и может генерироваться любой
    внешней расчётной программой
  \item Можно связать генерацию файла данных и компиляцию
    LaTeX-документа, сократив количество ручных операций
  \end{itemize}
\end{frame}

\begin{frame}[fragile]
  \frametitle{Пример: чтение файла, трёхмерные данные}
  \begin{figure}[!h]
    \centering
    \begin{tikzpicture}
      \begin{axis}[height=5cm,
        view={0}{90}, xlabel=$x$, ylabel=$t$, ticks=none,
        colormap/blackwhite, colorbar]
        \addplot3[surf] file {result.txt};
      \end{axis}
    \end{tikzpicture}
  \end{figure}
  \begin{itemize}
  \item 
\begin{lstlisting}[linewidth=.8\textwidth]
    \begin{axis}[view={0}{90},
      xlabel=$x$, ylabel=$t$,
      ticks=none, colormap/blackwhite, colorbar]
      \addplot3[surf] file {result.txt};
    \end{axis}
\end{lstlisting}
  \end{itemize}
\end{frame}

\begin{frame}
  \frametitle{Просмотр реальных примеров}
  \begin{itemize}
  \item Связь документа и расчётной программы позволяет сразу получать
    отчёт о работе, который достоверен и отражает настоящие результаты
  \end{itemize}
\end{frame}

\end{document}
