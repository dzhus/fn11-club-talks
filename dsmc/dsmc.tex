\documentclass[onlymath]{beamer}
\usepackage{fontspec}
\usepackage{xunicode,xltxtra}
\usepackage{polyglossia}
\usepackage{xltxtra}
\usepackage{fancybox}
\usepackage{graphicx}
\usepackage{tikz}
\usepackage{amsmath}
\usetikzlibrary{calc}
\usetikzlibrary{intersections}
\usetikzlibrary{decorations.pathmorphing}
\usetikzlibrary{decorations.pathreplacing}
\usetikzlibrary{patterns}
\usetikzlibrary{shapes.geometric}
\usetikzlibrary{positioning}
\tikzset{dot/.style={circle,fill=black,scale=0.5}}
\tikzset{particle/.style={circle,fill=black,scale=0.5}}
\usepackage[auto]{contour}
\contourlength{1pt}
\contournumber{8}

\usepackage{listings}
\lstset{language=haskell, basicstyle=\scriptsize\ttfamily, escapechar=|, frame=single}

\newcommand{\abs}[1]{\left \lvert{#1}\right \rvert}
\newcommand\tw\textwidth
\newcommand\neword\emph
\newcommand\code\texttt

\newcommand\avg[1]{\left\langle{#1}\right\rangle}

\renewcommand\geq\geqslant
\renewcommand\leq\leqslant
\newcommand{\pardiff}[2]{\frac{\partial{#1}}{\partial{#2}}}

\DeclareMathOperator\Kn{Kn}
\DeclareMathOperator\diverg{div}
\DeclareMathOperator\gradient{grad}

\renewcommand\epsilon\varepsilon
\newcommand{\cenfig}[2]{\begin{figure}\centering\includegraphics[scale=#1]{#2}
  \end{figure}}

\newcommand\good{{\color{green!50!black} \textbf{+}}}
\newcommand\bad{{\color{red!50!black} \textbf{--}}}
  
\usefonttheme{professionalfonts}
\usetheme[secheader]{Boadilla}
\usecolortheme{whale}

\setsansfont[Mapping=tex-text]{Myriad Pro}
\setmonofont[Mapping=tex-text]{DejaVu Sans Mono}

\setmainlanguage{russian}

\title{Прямое статистическое моделирование}
\author{Дмитрий Джус}
\institute{АК-121}
\date{25 апреля 2011}

\begin{document}

\begin{frame}
  \titlepage
\end{frame}

\begin{frame}
  \frametitle{План}
  \tableofcontents
\end{frame}

\section{Кинетическая теория газа}

\begin{frame}
  \frametitle{Простая модель одноатомного газа}
  \begin{itemize}
  \item Трёхмерное пространство с молекулами
  \item Твёрдые сферы постоянного диаметра $d$ и массы $m$, обладающие
    положением $\vec{x}(t)$ и скоростью $\vec{c}(t)$
  \item Положения и скорости меняются от взаимодействия частиц между
    собой и действия внешних сил
  \item Применяя методы \emph{классической механики}, зная
    $\vec{x}(t_0)$, $\vec{c}(t_0)$ для каждой частицы, можно найти
    зависимости $\vec{x}(t)$, $\vec{c}(t)$ (метод \emph{молекулярной
      динамики})
  \end{itemize}
\end{frame}

\begin{frame}
  \frametitle{Макроскопические параметры}
  \begin{itemize}
  \item Микроскопические параметры отдельных молекул интересны чуть
    менее, чем никак
  \item Макроскопические параметры \emph{осредняют} микроскопические
    по элементарным объёмам
  \item Микроскопические параметры связаны с молекулами,
    макроскопические — с точкой в пространстве
  \end{itemize}
\end{frame}

\begin{frame}
  \frametitle{Макроскопические параметры (2)}
  \begin{itemize}
  \item Концентрация $n$ — количество молекул в единице объёма
    \begin{equation*}
      n = \frac{N}{V}
    \end{equation*}
  \item Плотность $\rho$ — масса на единицу объёма
    \begin{equation*}
      \rho = m n
    \end{equation*}
  \item Скорость среды $\vec{V}=\avg{\vec{c}}$ — осреднение по скоростям молекул в
    элементарном объёме
    \begin{equation*}
      \avg{\vec{c}} = \frac{1}{N}\sum_{i \in V}{\vec{c}}
    \end{equation*}
  \item Температура $T$
    \begin{equation*}
      \frac{3kT}{m} = \avg{c^2}
    \end{equation*}
  \end{itemize}
\end{frame}

\begin{frame}
  \frametitle{Макроскопические параметры (3)}
  \begin{itemize}
  \item Компоненты тензора давления
    \begin{equation*}
      p_{ij} = \rho \avg{c_i} \avg{c_j}
    \end{equation*}
  \item Скалярный параметр давления в среде
    \begin{equation*}
      p = \frac{1}{3}\rho\avg{c^2}
    \end{equation*}
  \end{itemize}
\end{frame}

\begin{frame}
  \frametitle{Функция распределения}
  \begin{itemize}
  \item Частиц в среде очень много — вычислительные сложности
  \item Будем рассматривать сразу \emph{ансамбли} частиц, описываемые
    \emph{статистической} функцией распределения
    \begin{equation*}
      f(\vec{x}, \vec{c}, t)
    \end{equation*}
  \item Задана в шестимерном фазовом пространстве
  \item По определению, количество частиц в элементе $d\vec{x}\,
    d\vec{c}$ равно
    \begin{equation*}
      f(\vec{x}, \vec{c}, t) \,d\vec{x}\, d\vec{c}
    \end{equation*}
    Столько частиц имеют положение в интервале от $\vec{x}$ до
    $\vec{x}+d\vec{x}$ и скорость в интервале от $\vec{c}$ до
    $\vec{c}+d\vec{c}$.
  \end{itemize}
\end{frame}

\begin{frame}
  \frametitle{Вычисление макроскопических параметров}
  \begin{itemize}
  \item Число частиц в единице физического объёма
    \begin{equation*}
      n(\vec{x}, t) = \int f(\vec{x}, \vec{c}, t)\,d\vec{c}
    \end{equation*}
    Интегририрование проводится по всем скоростям.
  \item По определению, средняя величина $\vec{\Phi}$ вычисляется с
    как момент функции распределения $f$
    \begin{equation*}
      \avg{\Phi} = \frac{\int{\Phi f \,d\vec{c}}}{n}
    \end{equation*}
  \end{itemize}
\end{frame}

\begin{frame}
  \frametitle{Уравнение Больцмана}
  \begin{itemize}
  \item Описывает эволюцию ансамблей частиц
    \begin{equation*}
      \underbrace{\pardiff{f}{t} +
        \vec{c} \pardiff{f}{\vec{x}}}_{\text{изменение в фазовом
          пространстве}} + 
      \underbrace{K(f)}_{\text{интеграл столкновений}} = 0
    \end{equation*}
  \item Интеграл столкновений содержит модель взаимодействия частиц
  \item Не учитываем внешние силы
  \end{itemize}
\end{frame}

\begin{frame}
  \frametitle{Интеграл столкновений}
  \begin{itemize}
  \item Часто используемый вид $K(f)$ для одноатомного газа
    \begin{equation*}
      \int\int\int(f'f'_1-f f_1) {c_r \sigma \,d\Omega\,d\vec{c_1}}
    \end{equation*}
    \begin{itemize}
    \item $f, f_1$ — функции распределения до столкновения
    \item $f', f'_1$ — после
    \item $c_r$ — модуль относительно скорости
    \item $\sigma$ — сечение столкновения ($\sigma = \pi d^2$)
    \end{itemize}
  \item Описывает столкновение двух классов молекул
  \end{itemize}
\end{frame}

\begin{frame}
  \frametitle{Распределение Максвелла—Больцмана}
  \begin{itemize}
  \item Решение уравнения Больцмана для равновесного газа
    \begin{equation*}
      f_{\vec{c}}(c_x, c_y, c_z) = \left(\frac{m}{2\pi k T}\right)^{3/2}
      \exp\left[ \frac{-m(c_x^2 + c_y^2 + c_z^2)}{2kT} \right]
    \end{equation*}
  \end{itemize}
\end{frame}

\begin{frame}
  \frametitle{Решение уравнения Больцмана}
  \begin{itemize}
  \item 6-мерное уравнение в частных производных
  \item Сложность в задании граничных условий
  \item Даже на модельных задач получаются интегро-дифференциальные
    уравнения, которые в итоге всё равно нужно решать численно
  \end{itemize}
\end{frame}

\begin{frame}
  \frametitle{Уравнения сплошной среды}
  
  \begin{itemize}
  \item Уравнения для моментов функции распределения
    $\int\Phi(\vec{c}) f \,d\vec{c}$ приводят к известным уравнениям
    сплошной среды
    \begin{equation*}
      \begin{cases}
        \frac{d\rho}{dt} = -\rho \diverg {\vec{V}},\\
        \frac{d\vec{V}}{dt} = -\frac{1}{\rho}\gradient{p} + \frac{\lambda
          + \mu}{\rho}\gradient\diverg\vec{V} + \nu\Delta\vec{V},\\
        \rho\frac{de}{dt} = -p\diverg\vec{V}+\sum_{i,j}{\Pi_{ij}\pardiff{V_i}{x_j}}-\diverg\vec{W}
      \end{cases}
    \end{equation*}
  \item Вектор потока тепла $\vec{W}$ и тензор напряжений связаны с
    $\vec{V}$ и $T$
  \end{itemize}
\end{frame}

\begin{frame}
  \frametitle{Число Кнудсена}

  \begin{itemize}
  \item Характеризует степень разреженности газового потока:
    \begin{equation*}
      \Kn = \frac{\lambda}{L}
    \end{equation*}
    \begin{itemize}
    \item $\lambda$ — длина свободного пробега частиц
    \item $L$ — характерный размер \emph{локальной} области
      неоднородности (градиента макропараметров):
      \begin{equation*}
        L = \frac{\rho}{d\rho / dx}
      \end{equation*}
    \end{itemize}

  \item При $\Kn \to 0$ (в практике $\Kn < 0.1$) действует основное
    предположение МСС о континуальности среды
  \item При $0.1 < \Kn < 1$ требуется уже кинетическое моделирование
  \item При $\Kn \gg 1$ (в практике $\Kn \geq 1$) столкновения молекул
    между собой не учитываются (свободномолекулярное течение)
  \end{itemize}
\end{frame}

\begin{frame}
  \frametitle{На распутье}
  \begin{itemize}
  \item Пройден путь от молекулярной динамики к уравнению Больцмана и
    к континуальной модели
  \item Как же делать расчёт в разреженных газах?
  \item Околоземная орбита, микроструктуры
  \end{itemize}
\end{frame}

\section{Прямое статистическое моделирование}
\begin{frame}
  \frametitle{Общие положения}
  \begin{itemize}
  \item Хотим смоделировать обтекание тела газовым потоком с $0.1 <
    \Kn < 1$
  \item Имитируем поведение частиц, описанное уравнением Больцмана
  \item Расщепление:
    \begin{itemize}
      \item бесстолкновительное движение, изменяющее их положение (лагранжев этап)
      \item столкновение между частицами, изменяющее их скорости (эйлеров этап)
    \end{itemize}
  \item Как молекулярная динамика, только взаимодействия учитываются
    стохастически
  \item Считаем, что взаимодействовать между собой могут лишь частицы,
    находящиеся в одной пространственной ячейке (потенциал
    взаимодействия ограничен)
  \end{itemize}
\end{frame}

\begin{frame}
  \frametitle{Шаги алгоритма}
  \begin{itemize}
  \item Разбиение прямоугольной расчётной области на ячейки (регулярная
    сетка)
  \item Первоначально в области нет молекул
  \item Итерация на каждом временном шаге $\Delta t$ состоит из
    нескольких этапов
    \begin{itemize}
    \item Розыгрыш новых частиц на границе расчётной области по
      вектору набегающего потока $\vec{V}_\infty$
    \item Движение частиц в пространстве
    \item Сортировка частиц по ячейкам
    \item Розыгрыш 
    \item Удаление частиц, вышедших за пределы расчётной области
    \end{itemize}
  \item Моделирование заканчивается при установлении режима течения
  \item По полученному распределению частиц в расчётной области
    вычисляются макропараметры потока в окрестности тела
  \end{itemize}
\end{frame}


\begin{frame}
  \frametitle{Движение молекул в пространстве}
  \begin{itemize}
  \item Положение каждой частицы на $i$-м шаге преобразуется по закону
    \begin{equation*}
      \vec{x}_i = \vec{x}_{i-1} + \vec{c}_{i-1} \Delta t
    \end{equation*}
  \end{itemize}
\end{frame}

\begin{frame}[fragile]
  \frametitle{Учёт граничных условий}
  \begin{itemize}
  \item Если оказалось, что частица столкнулась с телом, возвращаем её
    в точку столкновения и разыгрываем столкновение с телом
  \item Для розыгрыша нужно знать вектор нормали $\vec{n}$ к телу в
    точке столкновения
  \item Тела сложные — что делать?
  \end{itemize}
\end{frame}

\begin{frame}
  \frametitle{Описание сложных тел}
  \begin{figure}
    \centering
    \includegraphics[scale=0.15]{csg.png}
  \end{figure}
  \begin{itemize}
  \item Используем подход CSG — конструктивная твердотельная геометрия
  \item Тело является рекурсивной композицией примитивов или других
    тел
    \begin{equation}
      B_\Sigma = (B_1 \cap B_2) \cup \overline{B_3}
    \end{equation}
  \item Примитивы — сфера, цилиндр, полупространство и другие
  \end{itemize}
\end{frame}

\begin{frame}
  \frametitle{Учёт столкновений со сложным телом}

\begin{itemize}
\item Вводим для частицы и тела функцию $\tau(P, B)$, которая
  возвращает набор отрезков на оси $t$, соответствующих участку
  траектории частицы внутри тела («след частицы на теле»)
  \end{itemize}
  \begin{figure}
    \centering
    \begin{tikzpicture}[scale=0.75]
      \begin{scope}
        \coordinate (O) at (0, 0);
        \coordinate (p) at ($ (O) + (120:2) $); 
        \draw[name path=body, draw] (O) to[out=90, in=180] ++(45:5)
        to[out=0, in=25] ++(-75:7)
        node[label=below:$B$] {}
        to[out=205, in=-10] ++(110:4)
        to[out=170, in=-90] (O);

        \draw[name path=ray, thick, densely dotted, ->] (p) -- ++(155:1) -- ++(-25:11);
        \begin{scope}[name intersections={of=body and ray, by={a,b,c,d}}]
          \draw[ultra thick] (a) -- (d);
          \draw[ultra thick] (c) -- (b);
          \node at ($(a) + (240:0.5)$) {\contour{white}{$(t_1, \vec{n}_1)$}};
          \node at ($(d) + (90:0.5)$) {\contour{white}{$(t_2, \vec{n}_2)$}};
          \node at ($(c) + (90:0.4)$) {\contour{white}{$(t_3, \vec{n}_3)$}};
          \node at ($(b) + (90:0.5)$) {\contour{white}{$(t_4, \vec{n}_4)$}};
          \node[particle, label=below:$P$] at (p) {};

          \draw[->] (a) -- ++(165:0.5);
          \draw[->] (d) -- ++(-70:0.5);
          \draw[->] (c) -- ++(179:0.5);
          \draw[->] (b) -- ++(-20:0.5);
        \end{scope}
      \end{scope}
    \end{tikzpicture}
\end{figure}
\end{frame}

\begin{frame}
  \frametitle{Расчёт точки столкновения}
  \begin{itemize}
  \item Для каждого примитива CSG-дерева строим след частицы на нём
    (уравнение от $t$)
  \item Объединяем следы по рекурсивным правилам
    \begin{itemize}
    \item $\tau(p, B_1 \cup B_2) = \tau(p, B_1) \cup \tau(p, B_2)$
    \item $\tau(p, B_1 \cap B_2) = \tau(p, B_1) \cap \tau(p, B_2)$
    \item $\tau(p, \overline{B}) = \overline{\tau(p, B)}$
    \end{itemize}
  \item Получаем итоговый след $\tau_\Sigma = \tau(p, B_\Sigma)$
  \item Если $\tau_\Sigma \cap (-\Delta t, 0) = (t_h, 0)$, то
    частица столкнулась с телом в момент времени $t_h < 0$, по
    которому можно определить и точку столкновения
  \end{itemize}
\end{frame}

\begin{frame}
  \frametitle{Столкновение молекул между собой}
  \begin{itemize}
  \item Учитываем только частицы, относящиеся к одной ячейке
    \begin{figure}
      \centering
      \includegraphics[scale=0.25]{cells.png}
    \end{figure}
  \item Для каждой пары частиц $i, j$ в ячейке разыгрываем вероятность
    столкновения
    \begin{equation*}
      P_{ij} = \frac{\sigma_{ij}\vec{c}_r\Delta t}{V_c}
    \end{equation*}
    \begin{itemize}
    \item $V_c$ — объём ячейки
    \item $\vec{c}_r$ — относительная скорость двух частиц
    \item $\sigma_{ij}$ — сечение столкновения, зависящее от
      используемой модели частиц
    \end{itemize}
  \end{itemize}
\end{frame}

\begin{frame}
  \frametitle{Что происходит в ячейке}
  \begin{figure}
    \centering
    \includegraphics[scale=0.25]{sigma.png}
  \end{figure}
  \begin{itemize}
  \item Ищем отношение объёма цилиндра высотой $\Delta t$ и площадью
    основания $\sigma_{ij}$ к объёму всей ячейки
  \item Испытания по схеме Бернулли — частицы сталкиваются, если
    \begin{equation*}
      P_{ij} > R[0,1]
    \end{equation*}
  \end{itemize}
\end{frame}

\begin{frame}
  \frametitle{Модели частиц}
  \begin{itemize}
  \item Определяет модель взаимодействия при столкновении частиц
  \item Модель твёрдых сфер: 
    \begin{itemize}
    \item $\sigma = \pi d^2$
    \item Вектор $\vec{c}_r$ после столкновения распределён по
      направлениями \emph{равномерно}
    \item Скорости после столкновения:
      \begin{equation*}
        \begin{cases}
          \vec{c}' = \vec{c} + [(\vec{c_1} - \vec{c})\vec{c}_r]\vec{c}_r,\\
          \vec{c}'_1 = \vec{c}_1 + [(\vec{c_1} - \vec{c})\vec{c}_r]\vec{c}_r
        \end{cases}
      \end{equation*}
    \end{itemize}
  \item Есть модели с учётом при взаимодействии внутренней энергии и
    другие
  \end{itemize}
\end{frame}

\begin{frame}
  \frametitle{Модели поверхности}
  \begin{itemize}
  \item Определяет взаимодействие частиц с поверхностью
  \item Модель зеркального отражения:
    \begin{itemize}
    \item Молекула отскакивает от поверхности, как бильярдный шарик
    \item $\vec{c}' = \vec{c} - 2\vec{n}(\vec{n} \cdot\vec{v})$
    \end{itemize}
  \item Есть модели диффузного отражения, с учётом температур и другие
  \end{itemize}
\end{frame}

\begin{frame}
  \frametitle{Вычисление макропараметров}
  \begin{itemize}
  \item Действуем по определению
  \item Разбиение области на ячейки, осреднение параметров частиц в
    каждой ячейке
  \item Результат — распределение макропараметров
  \end{itemize}
\end{frame}

\begin{frame}
  \frametitle{Параллелизм}
  \begin{itemize}
  \item \good{} Движение молекул
  \item \good{} Столкновения в разных ячейках
  \item Нужны некорреллированные генераторы случайных чисел на каждую ячейку
  \item \good{} Расчёт $\tau$ для разных молекул (можно использовать
    GPU, так как используются лишь простые векторные операции)
  \end{itemize}
\end{frame}
\end{document}
