\documentclass[onlymath]{beamer}
\usepackage{fontspec}
\usepackage{xunicode,xltxtra}
\usepackage{polyglossia}
\usepackage{xltxtra}
\usepackage{fancybox}
\usepackage{graphicx}
\usepackage{tikz}
\usepackage{amsmath}
\usetikzlibrary{calc}
\usetikzlibrary{intersections}
\usetikzlibrary{decorations.pathmorphing}
\usetikzlibrary{decorations.pathreplacing}
\usetikzlibrary{patterns}
\usetikzlibrary{shapes.geometric}
\usetikzlibrary{positioning}
\tikzset{dot/.style={circle,fill=black,scale=0.5}}
\tikzset{particle/.style={circle,fill=black,scale=0.5}}
\usepackage[auto]{contour}
\contourlength{1pt}
\contournumber{8}

\usepackage{pgfplots}
\usepackage{listings}
\lstset{language=haskell, basicstyle=\scriptsize\ttfamily, escapechar=|, frame=single}


\newcommand{\abs}[1]{\left \lvert{#1}\right \rvert}
\newcommand\tw\textwidth
\newcommand\neword\emph
\newcommand\code\texttt

\newcommand\avg[1]{\left\langle{#1}\right\rangle}

\renewcommand\geq\geqslant
\renewcommand\leq\leqslant
\newcommand{\pardiff}[2]{\frac{\partial{#1}}{\partial{#2}}}

\DeclareMathOperator\Kn{Kn}
\DeclareMathOperator\diverg{div}
\DeclareMathOperator\gradient{grad}

\renewcommand\epsilon\varepsilon
\newcommand{\cenfig}[2]{\begin{figure}\centering\includegraphics[scale=#1]{#2}
  \end{figure}}

\newcommand\good{{\color{green!50!black} \textbf{+}}}
\newcommand\bad{{\color{red!50!black} \textbf{--}}}

\usefonttheme{professionalfonts}
\usetheme[secheader]{Boadilla}
\usecolortheme{whale}

\setsansfont[Mapping=tex-text]{Myriad Pro}
\setmonofont[Mapping=tex-text]{DejaVu Sans Mono}

\setmainlanguage{russian}

\newcommand{\tr}\tau
% time step
\newcommand{\ts}{\Delta t}
% cell step
\newcommand{\sps}{\Delta L}
% weight
\newcommand{\statw}{W_N}
% domain
\newcommand\dom{\Omega}


% Kinetic parameters
\newcommand{\fp}{\lambda_m}
\newcommand{\ft}{\tau_m}
\newcommand{\cf}{\nu_m}
\newcommand{\ttr}{T_{\rtr}}
\newcommand{\dcs}{\sigma}
\newcommand{\ccs}{\sigma_T}
\newcommand{\mff}{\avg{\ccs c_r}_{\max}}
\newcommand\then{\Rightarrow}

% Macro parameters
\newcommand\flowvel{\vec{V}_{\infty}}
\newcommand\flowtemp{T_{\infty}}
\newcommand\flowcon{n_{\infty}}

\newcommand{\floor}[1]{\left \lfloor{#1}\right \rfloor}

% Velocities
\newcommand\cprob{c'_p}

\title{Моделирование разреженных газовых потоков методом Монте-Карло}
\author{Дмитрий Джус}

\begin{document}

\begin{frame}
  \titlepage
\end{frame}

\begin{frame}
  \frametitle{Концептуальная постановка}
  Рассматривается обтекание тела сложной формы $B$ разреженным потоком
  одноатомного газа с заданными параметрами $\flowvel, \flowtemp,
  \flowcon$ в области $\dom$ в режиме $0.1 < \Kn < 10.0$, $\Kn = \fp /
  L$.
  \begin{figure}
    \centering
    \includegraphics[scale=0.2]{concept1.png}
  \end{figure}
  Газовый поток моделируется как ансамбль частиц, у каждой из которых
  есть положение $\vec{x}$ и скорость $\vec{c}$. Требуется найти
  распределение макроскопических параметров в окрестности тела.
\end{frame}

\begin{frame}
  \frametitle{Математическая постановка}
  В отсутствие условия $\Kn \ll 1$ рассматриваем модель идеального
  газа. Учитываются парные столкновения.
  
  Состояние газа описывается уравнением Больцмана для функции
  распределения $f(\vec{x}, \vec{c}, t)$:
\begin{equation*}
  \label{eq:boltzmann}
  \pardiff{f}{t}
  +\underbrace{\vec{c}\pardiff{f}{\vec{x}}}_{\text{конвекция}}
  +\underbrace{\vec{F}\pardiff{f}{\vec{c}}}_{\text{действие внешних сил}}
  = \underbrace{K(f)}_{\text{столкновения}}
\end{equation*}
Для решения требуются краевые условия.
\begin{equation}
  \label{eq:coll-operator}
  K(f) = \int_{-\infty}^{+\infty}\int_{0}^{4\pi}{n^2(f'f'_1 - f
  f_1)\dcs c_r\,d\Omega\,d\vec{c}_1},
\end{equation}

где $f, f_1$ — функции распределения молекул со скоростями $\vec{c},
\vec{c}_1$ до столкновения, $f', f'_1$ — функции распределения молекул
со скоростями $\vec{c}', \vec{c}'_1$, $\dcs$ — дифференциальное
сечение столкновения, $\Omega$ — телесный угол, $c_r$ — относительная
скорость.
\end{frame}

\begin{frame}
  \frametitle{Основные предположения метода прямого статистического моделирования}
  \begin{enumerate}
\item Описываемое уравнением Больцмана молекулярное движение в
  пространстве и столкновения частиц между собой расщеплены по
  времени. В пределах каждой итерации эти два этапа (называемые
  «лагранжев этап» и «эйлеров этап», соответственно) рассматриваются
  раздельно.
\item Каждая модельная частица-симулятор представляет большое
  количество $\statw$ настоящих частиц.
\item Расчётная область разбивается на пространственные ячейки, и
  считается, что на каждой итерации между собой сталкиваются лишь
  частицы, относящиеся к одной и той же ячейке. Расчёт столкновений
  между частицами проводится стохастически, без учёта их реального
  взаимоположения.
\end{enumerate}
\end{frame}

\begin{frame}
  \frametitle{Расщепление уравнения Больцмана по времени}
  Вводится схема расщепления уравнения Больцмана по времени с шагом
  $\ts$. Рассматривается два набора функций $f_k'(\vec{x},\vec{c},t)$
  и $f_k''(\vec{x},\vec{c},t)$, заданных на временных интервалах
  $[t_k; t_{k+1}]$, где $t_k = k\ts$. Первое уравнение для $f_k'$
  соответствует бесстолкновительному движению частиц в пространстве
  (лагранжев этап):
  \begin{equation*}
    \label{eq:dsmc-splitting1}
    \pardiff{f_k'}{t} + \vec{c}\pardiff{f_k'}{\vec{x}} = 0.
  \end{equation*}
  Уравнение для $f_k''$ описывает только процесс столкновений частиц
  между собой без пространственной конвекции (эйлеров этап):
  \begin{equation*}
    \label{eq:dsmc-splitting2}
    \pardiff{f_k''}{t} = K(f).
  \end{equation*}
Правило чередования этапов:
\begin{equation*}
  \label{eq:dsmc-splitting3}
  \begin{aligned}
    f_k''(\vec{x},\vec{c},t_k) &=
    f_{k-1}''(\vec{x},\vec{c},t_k),\\
    f_k'(\vec{x},\vec{c},t_k) &=
    f_{k-1}''(\vec{x},\vec{c},t_k),\:k=1,2,\dotsc;\:
    f_0'(\vec{x},\vec{c},t_k) &= f_0(\vec{x}, \vec{c}, 0).
  \end{aligned}
\end{equation*}
\end{frame}

\begin{frame}
  \frametitle{Расщепление столкновений по пространству}
  Основные трудности при решении уравнения Больцмана связаны с
  интегральным видом $K(f)$. Область $\dom$ разбивается на $C$ ячеек,
  в каждой из которых на всякой итерации оказывается некоторое
  количество частиц $N_C$.
  \begin{figure}[!h]
    \centering
    \includegraphics[scale=0.2]{grid.png}
    \includegraphics[scale=0.2]{cells.png}
  \end{figure}
\end{frame}

\begin{frame}
  \frametitle{Расщепление столкновений по пространству}
  В каждой ячейке в течение итерации рассматривается скачкообразный
  марковский процесс для $3N_C$-мерной системы из $N_C$ векторов
  скоростей частиц в данной ячейке
  \begin{equation*}
    \vec{C} = (\vec{c}_1, \dotsc, \vec{c}_{N_C}).
  \end{equation*}
  Во время скачка меняются только две скорости (бинарные
  столкновения):
  \begin{equation*}
    \vec{C} = (\vec{c}_1,\dotsc,\vec{c}_i,\dotsc,\vec{c}_j,\dotsc,\vec{c}_{N_C})
  \end{equation*}
  переходит в состояние 
  \begin{equation*}
    \vec{C'} = (\vec{c}_1,\dotsc,\vec{c'}_i,\dotsc,\vec{c'}_j,\dotsc,\vec{c}_{N_C}).
  \end{equation*}
  Новые значения $\vec{c'}_i$ и $\vec{c'}_j$ являются случайными
  величинами, удовлетворяющими законам сохранения:
  \begin{equation*}
    \begin{aligned}
      \frac{\vec{c'}_i+\vec{c'}_j}{2} &=
      \frac{\vec{c}_i+\vec{c}_j}{2},\\
      c_r = \abs{c_i - c_j} &= \abs{c'_i - c'_j} = c'_r.
    \end{aligned}
  \end{equation*}
 Вектор $\vec{c'}_r = \vec{c'}_i - \vec{c'}_j$ распределён по
  телесному углу $\Omega$ с заданным законом распределения $\dcs$
  (см. \eqref{eq:diff-cross-section}).
\end{frame}

\begin{frame}
  \frametitle{Расщепление столкновений по пространству}
  Вероятность перехода системы $\vec{C} \to \vec{C'}$ и время ожидания
  между переходами $\tau$ задаются следующим образом:
  \begin{equation*}
    \label{eq:coll-prob-generic}
    P_{ij} = \frac{\statw\ccs^{(i,j)}c_r^{(i, j)} \ts}{V_C},\:    F(\tau) = 1 - \exp(-\cf \tau).
  \end{equation*}
  \begin{figure}[!h]
    \centering
    \includegraphics[scale=0.1]{coll.png}
  \end{figure}
  $\cf = \avg{\ccs c_r} n$ — частота столкновений, $\statw$ —
  статистический вес частиц.
\end{frame}

\begin{frame}
  \frametitle{Статистический вес частиц}
  $\statw$ — статистический вес одной частицы:
  \begin{equation*}
    \ccs n = \ccs' n' \then \cf = \avg{\ccs c_r} \cdot n = \avg{\ccs' c_r} \cdot n'.
  \end{equation*}
  При увеличении $\statw$ и пропорциональном уменьшении $n$ частота
  $\cf$ не меняется, поэтому можно моделировать поток меньшим
  количеством частиц-«симуляторов».
  \begin{equation*}
    n = 10^{20}, W_N = 10^{13} \then n_s = 10^7.
  \end{equation*}
\end{frame}

\begin{frame}
  \frametitle{Стохастический расчёт столкновений}
  При столкновениях $\vec{c'}_r$ распределён согласно заданному закону
  $\sigma$ — дифференциальному сечению столкновения (следствие теории
  взаимодействия молекул):
  \begin{equation*}
    \sigma(\chi,c_r) d\Omega = b\,db\,d\epsilon
  \end{equation*}

  В модели твёрдых сфер постоянного диаметра $d$
  \begin{equation*}
    \sigma = d^2/4,
  \end{equation*}
  Так что $\vec{c'}_r$ — изотропный случайный вектор.
  \begin{figure}[!h]
    \centering
    \includegraphics[scale=0.1]{scatter.png}
  \end{figure}
\end{frame}

\begin{frame}
  \frametitle{Граничные и начальные условия}
  Для условия на свободных границах $\dom$ используется распределение
  Максвелла—Больцмана (невозмущённая среда в окрестности $\dom$):
  \begin{equation*}
    \label{eq:maxwell-boltzmann}
    f_M = f_{\vec{c}}(c_1, c_2, c_3) = \left(\frac{m}{2\pi k T}\right)^{3/2}
    \exp\left(-\frac{m(c_1^2 + c_2^2 +c_3^2)}{2kT}\right)
  \end{equation*}
  \begin{equation*}
    \label{eq:maxwell-boltzmann-component}
    f_{c_i} = \frac{\beta}{\sqrt{\pi}}\exp(-\beta^2{c_i}^2),\:  \beta = \sqrt\frac{m}{2kT}
  \end{equation*}
  
  Начальное распределение $f(\vec{x}, \vec{c}, 0)$: $\dom$ пуста, либо
  $f = f_M$.
  
  Взаимодействие с поверхностью тела описывается ядром столкновения
  $R(\vec{c}_0, \vec{c}_r)$:
  \begin{equation*}
    f^+(\vec{c}_r) \vec{c}_r = -\int_{\vec{c}_o \cdot
      \vec{n} < 0}{R(\vec{c}_0, \vec{c}_r)}\vec{c}_0 \cdot
    f^-(\vec{c}_0)\,d\vec{c}_0,\quad(\vec{c}_r\cdot\vec{n}>0).
  \end{equation*}
  $\vec{c}_r$ — скорость отражённой молекулы, $\vec{c}_0$ — скорость
  падающей молекулы, $f^+$ — функция распределения по скоростям для
  отражённых молекул, $f^-$ — функция распределения для падающих
  молекул.
\end{frame}

\begin{frame}
  \frametitle{Граничные условия на тело}
  Диффузное отражение ($T_w$ — температура стенки):
  \begin{equation*}
    \begin{aligned}
      R_D(\vec{c}_0, \vec{c}_r) &=
      \frac{1}{2\pi}\left(\frac{m}{kT_w}\right)^2\exp\left(-\frac{m}{2kT_w}c^2_r\right),\\
      f^+(\vec{c}_r) &= \left(\frac{m}{2\pi k
          T_w}\right)^{3/2}\exp\left(-\frac{m}{2kT_w}c^2_r\right).
    \end{aligned}
  \end{equation*}
  Модель Черчиньяни—Лемпис—Лорда ($\sigma_{\tau}, \alpha_n$ —
  коэффициенты модели):
  \begin{equation*}
    \label{eq:cll-tau}
    R_{C_{\tau}}(v_i, v_r) = \frac{1}{\sqrt{\pi \sigma_{\tau} (2-\sigma_{\tau})}}
    \exp{\left(-\frac{(v_r-(1-\sigma_{\tau})v_i)^2}{\sigma_{\tau}(2-\sigma_{\tau})}\right)}.
  \end{equation*} — для касательных компонент $\vec{c}$.
  \begin{equation*}
    R_{C_{n}}(u_i, u_r) = \frac{2 u_r}{\alpha_n}
    \exp{\left(-\frac{u^2_r+(1-\alpha_n)u^2_i}{\alpha_n}\right)} \cdot
    I_0\left(\frac{2 u_r u_i \sqrt{1-\alpha_n}}{\alpha_n}\right),
  \end{equation*} — для нормальной компоненты.
\end{frame}

\begin{frame}
  \frametitle{Граничные условия на тело (2)}
  \begin{figure}[!h]
    \centering
    \includegraphics[scale=0.2]{boundary.png}
  \end{figure}
\end{frame}

\begin{frame}
  \frametitle{Блок-схема ПСМ}
  Метод прямого статистического моделирования имитирует процессы,
  проходящие в газе, согласно законам распределения соответствующих
  случайных величин.
  \begin{figure}[!h]
    \centering
    \includegraphics[scale=0.3]{algo.png}
  \end{figure}  
\end{frame}

\begin{frame}
  \frametitle{Расчётная область}
  \begin{figure}[!h]
    \centering
    \includegraphics[scale=0.2]{domain.png}
  \end{figure}   
\end{frame}

\begin{frame}
  \frametitle{Инъекция частиц на границе $\dom$ — метод объёмных
    источников}
  Для добавления частиц в систему используется шесть вспомогательных
  подобластей числу граней $\dom$:
  \begin{figure}[!h]
    \centering
    \includegraphics[scale=0.21]{sources.png}
  \end{figure}    
\end{frame}

\begin{frame}
  \frametitle{Инъекция частиц на границе $\dom$}
  В каждой $\dom_i$ разыгрывается $N_i = \flowcon V_i / W_N$ частиц,
  распределённых в пространстве равномерно, а по скоростям — согласно
  $f_M$:
 \begin{equation*}
    f_{c_i} = \frac{\beta}{\sqrt{\pi}}\exp(-\beta^2{c_i}^2),\:  \beta = \sqrt\frac{m}{2kT}
  \end{equation*}
  \begin{figure}[!h]
    \centering
    \includegraphics[scale=0.2]{sources2.png}
  \end{figure}    
  Новые молекулы добавляются к системе рассматриваемых частиц.
\end{frame}

\begin{frame}
  \frametitle{Удаление частиц}
  После лагранжева этапа в системе остаются только те частицы, которые
  лежат в основной области $\dom$:
  \begin{equation*}
    (x > x_{\min}) \land (x < x_{\max}) \land (y > y_{\min}) \land
    (y < y_{\max}) \land (z > z_{\min}) \land (z < z_{\max}),
  \end{equation*}
  Остальные частицы удаляются.
\end{frame}

\begin{frame}
  \frametitle{Установление потока}
  Поток считается установившимся, если выполнен критерий
    стабилизации:
    \begin{equation*}
      \abs{\frac{N}{N^-}} < \epsilon_N,
    \end{equation*}
    $N$ — число частиц на текущей итерации, $N^-$ — число частиц на
    предыдущей итерации.
\end{frame}

\begin{frame}
  \frametitle{Установление потока (2)}
  \begin{figure}[!h]
    \begin{tikzpicture}
      \begin{axis}[ylabel=$N$, xlabel=$t$, y=0.0000999]
        \addplot file{steady.txt};
      \end{axis}
    \end{tikzpicture}
  \end{figure}
\end{frame}

\begin{frame}
  \frametitle{Расчёт макропараметров в ячейке}
  \begin{equation*}
    \begin{aligned}
        \vec{V} &= \avg{\vec{c}} = 
        \frac{1}{N_C}\sum_{i = 1}^{N_C}{\vec{c}_i},\\
        \rho &= nm = \frac{N_C}{V_C}\statw m,\\
        p &= \frac{1}{3}\rho\avg{c'^2} = 
        \frac{1}{3}\rho\sum_{i=1}^{N_C}(\vec{c}_i - \vec{V})^2,
      \end{aligned}
    \end{equation*}
    $N_C$ — количество частиц в ячейке, $V_C$ — объём ячейки.
\end{frame}

\begin{frame}
  \frametitle{Выбор $\ts$, $\sps$, $\statw$}
  Для газа в невозмущённом потоке определяются средние частота
  столкновений $\cf$, длина свободного пробега $\fp$ и время между
  столкновениями $\ft$:
    \begin{equation*}
      \begin{aligned}
        \cf &= \avg{\ccs c_r} \flowcon,\:\fp = \frac{\avg{c'}}{\cf},\:\ft = {\cf}^{-1}\\
        \fp &= \frac{1}{\sqrt{2} \pi d^2 \flowcon} \then \sps < \fp,\\
        \ft &= \frac{1}{2d^2 \flowcon}\sqrt{\frac{m}{\pi k T}} \then \ts < \ft.
      \end{aligned}
    \end{equation*}
  
    Критерий выбора веса частиц $\statw$:
    \begin{equation*}
      \avg{N_C}\cdot\Kn_C \gg 1,
    \end{equation*}
    где $\Kn_C = \frac{\fp}{\sps}$, $\avg{N_C} = \frac{\flowcon V_C}{W_N}$.
\end{frame}


\begin{frame}
  \frametitle{Верификация}
  Обтекание цилиндра $\flowcon = 10^{21}, \flowtemp = 100K, \flowvel =
  (1000, 0, 0), \Kn = 0.05, T_w = 300K$. Поле плотностей перед
  цилиндром и линии уровня:
  \begin{figure}
    \centering
    \includegraphics[scale=0.3]{verify1.png}
    \includegraphics[scale=0.3]{verify11.png}
  \end{figure}
\end{frame}


\begin{frame}
  \frametitle{Верификация (2)}
  Результаты Бёрда были получены с помощью программы для
  осесимметричных сечений. В нашей постановке пришлось усечь цилиндр
  плоскостями до четверти. Поле модуля скорости:
  \begin{figure}
    \centering
    \includegraphics[scale=0.3]{verify2.png}
  \end{figure}
\end{frame}

\begin{frame}
  \frametitle{Результаты}
  \begin{itemize}
  \item Геометрия на рисунке построена в Paraview
  \item Сложные тела, аргон, $T = 300K, \vec{V} = (1200, 0, 0), m =
    21, n = 10^{21}, W = 10^{15}, \Delta t = 10^{-6}$
  \end{itemize}
  \begin{figure}
    \centering
    \includegraphics[scale=0.2]{b1.png}
  \end{figure}
\end{frame}

\begin{frame}
  \frametitle{Результаты (2)}
  \begin{figure}
    \centering
    \includegraphics[scale=0.2]{b2.png}
  \end{figure}
\end{frame}

\begin{frame}
  \frametitle{Результаты (3)}
  \begin{figure}
    \centering
    \includegraphics[scale=0.2]{b3.png}
  \end{figure}
\end{frame}

\begin{frame}
  \frametitle{Результаты (4)}
  \begin{figure}
    \centering
    \includegraphics[scale=0.2]{b4.png}
  \end{figure}
\end{frame}

\end{document}
