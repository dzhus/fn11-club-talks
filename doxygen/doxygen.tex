\documentclass[onlymath]{beamer}
\usepackage{fontspec}
\usepackage{xunicode,xltxtra}
\usepackage{polyglossia}
\usepackage{xltxtra}
\usepackage{fancybox}
\usepackage{graphicx}

\usepackage{listings}
\lstset{language=c++, basicstyle=\scriptsize\ttfamily, escapechar=|, frame=single}

\newcommand{\abs}[1]{\left \lvert{#1}\right \rvert}
\newcommand\tw\textwidth
\newcommand\neword\emph
\newcommand\code\texttt
\renewcommand\epsilon\varepsilon
\newcommand{\cenfig}[2]{\begin{figure}\centering\includegraphics[scale=#1]{#2}
  \end{figure}}
\newcommand\good{{\color{green!50!black} \textbf{+}}}
\newcommand\bad{{\color{red!50!black} \textbf{--}}}
  
\usefonttheme{professionalfonts}
\usetheme[secheader]{Boadilla}
\usecolortheme{whale}

\setsansfont[Mapping=tex-text]{Myriad Pro}
\setmonofont[Mapping=tex-text]{DejaVu Sans Mono}
\setmainlanguage{russian}

\title{Документирование кода и Doxygen}
\author{Дмитрий Джус}
\institute{АК-101}
\date{27 апреля 2011}

\begin{document}

\begin{frame}
  \titlepage
\end{frame}

\begin{frame}
  \frametitle{План}
  \tableofcontents
\end{frame}

\section{Общие вопросы}

\begin{frame}[fragile]
  \frametitle{Зачем документировать код}
  \begin{itemize}
  \item Код приходится читать гораздо чаще, чем писать

  \item Работа в коллективе предусматривает отсутствие скрытых знаний

  \item Программируем осмысленно

  \item Помимо сопроводительной записки к библиотеке, простейшая
    документация — комментарии в коде
  \end{itemize}
\end{frame}

\begin{frame}[fragile]
  \frametitle{Основные принципы}
  \begin{itemize}
  \item Документация и код — разные представления одной сущности

  \item Не повторяйте очевидное.

    Бесполезные комментарии только вредят:
\begin{lstlisting}
/// Increase i
i++;
\end{lstlisting}
    
  \item Важно поддерживать актуальность описаний

  \item У хорошего кода хорошая документация
  \end{itemize}
\end{frame}

\begin{frame}
  \frametitle{Жизнь и смерть}
  \begin{itemize}
  \item Код живёт, пока с ним кто-то может работать
  \item Как работать с непонятным кодом?
  \end{itemize}
\end{frame}

\begin{frame}
  \begin{itemize}
    \frametitle{Простой текст}
    
  \item Рукописные заметки в сторонних текстовых файлах
    (\texttt{README}, \texttt{/usr/src/linux/Documentation})
    
  \item \good{} Просто и быстро писать
  \item \bad{} Отсутствие структурирования
  \item \bad{} Может устареть    
  \end{itemize}
\end{frame}

\begin{frame}
  \frametitle{Структурные форматы}
  \begin{itemize}
  \item Какой-либо специальный формат для описания документации
    (LaTeX, DocBook, Texinfo и другие)
  \item Используются и заточенные под конкретный проект XML-языки
    (Gentoo Guide-XML)
  \item Экспорт в разные форматы представления (HTML, PDF, info и
    прочие)

  \item \good{} Структурность
  \item \bad{} Может устареть
  \item \bad{} Неудобно писать
  \end{itemize}
\end{frame}

\section{Doxygen}

\begin{frame}
  \frametitle{Документация из комментариев}
  \begin{itemize}
  \item Всё началось с Javadoc, теперь есть Doxygen, gtk-doc и другие
  \item Основная идея: внешняя программа обрабатывает исходные тексты
    программы, генерируя по ним документацию
  \item В комментариях помимо обычного текста используются конструкции
    со специальным значением
  \item \good{} Пишем код и документацию одновременно
  \item \good{} Обработка автоматизирована
  \item \good{} Поддерживаются разные языки программирования
  \item \good{} Несколько форматов вывода: HTML, PDF, man, XML
  \end{itemize}
\end{frame}

\begin{frame}[fragile]
  \frametitle{Простой пример}
  \begin{itemize}
  \item Doxygen извлекает информацию о функциях, классах и методах, а
    также комментарии вида
    \begin{lstlisting}
      /// comment
      /** comment */
    \end{lstlisting}
    \begin{lstlisting}
      /// Discrete Fourier transform for N points
      void dft(float *x_real, float *x_imag, const int N);
    \end{lstlisting}
  \end{itemize}
\end{frame}

\begin{frame}[fragile]
  \frametitle{Как пользоваться}
  \begin{itemize}
  \item Создаём в корне проекта конфигурационный файл (\texttt{Doxyfile})
\begin{lstlisting}
doxygen -g
\end{lstlisting}
  \item Изменяем настройки в \texttt{Doxyfile}
  \item Запускаем doxygen
\begin{lstlisting}
doxygen
\end{lstlisting}
  \item Смотрим получившуюся документацию
  \end{itemize}
\end{frame}

\begin{frame}[fragile]
  \frametitle{Ссылки}
  \begin{itemize}
  \item Doxygen обрабатывает всё дерево проекта и может создавать
    перекрестные ссылки
    \begin{lstlisting}
      /// Fast Fourier transform for N points
      /// @see dft
      void fft(float *x_real, float *x_imag, const int N);
    \end{lstlisting}
  \end{itemize}
\end{frame}

\begin{frame}[fragile]
  \frametitle{Документация методов}
  \begin{itemize}
  \item Ключевое слово \texttt{@param} задаёт описание аргумента метода
    \begin{lstlisting}
/// Fast Fourier transform for N points
///
/// @param x_real Real parts
/// @param x_image Imaginary parts
/// @param N Points in sample
void fft(float *x_real, float *x_imag, const int N);
    \end{lstlisting}
  \item Ключевое слово \texttt{@return} ставится перед описанием
    возвращаемого значения, а \texttt{@throws} — перед вызываемым
    исключением
\begin{lstlisting}
/// @return Amount of prime numbers in array
/// @throws PrimeCountException
int count_primes(int *n, const int len);
\end{lstlisting}
  \end{itemize}
\end{frame}

\begin{frame}[fragile]
  \frametitle{Списки задач}
  \begin{itemize}
  \item Ключевые слова \texttt{@todo} и \texttt{@bug} позволяют
    отметить существующие проблемы в коде
\begin{lstlisting}
/// @todo Rewrite using Miller-Rabin algorithm
int count_primes(int *n, const int len);
\end{lstlisting}
\begin{lstlisting}
/// @bug Fails when no primes are present
int count_primes(int *n, const int len);
\end{lstlisting}
  \item Опции \texttt{GENERATE\_TODOLIST} и \texttt{GENERATE\_BUGLIST}
    управляют генерацией списков задач или проблем по коду
  \end{itemize}
\end{frame}

\begin{frame}
  \frametitle{Генерация графов зависимостей}
  \begin{itemize}
  \item Опции \texttt{HAVE\_DOT}, \texttt{CLASS\_GRAPH},
    \texttt{CALL\_GRAPH}, \texttt{CALLER\_GRAPH},
    \texttt{GRAPHICAL\_HIERARCHY} управляют генерацией графов
    зависимостей между классами и вызовов функций
  \item Используется утилита \texttt{dot} из пакета Graphviz
  \item Ключевое слово @dot позволяет вставить вручную описание графа
  \end{itemize}
\end{frame}

\begin{frame}
  \frametitle{Многие другие возможности}
  \begin{itemize}
  \item Индексы по имеющимся в проекте методам, классам
  \item Форматирование текста документации (\texttt{@b}, \texttt{@c},
    \texttt{@e}, а также подмножество HTML)
  \item HTML-шаблоны настраиваются
  \item Условная вставка документации в зависимости от выходного формата
  \end{itemize}
\end{frame}

\end{document}
